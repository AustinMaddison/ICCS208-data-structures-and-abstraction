\chapter{Task 6: Recursive Code}

\subsection*{P1}
\begin{lstlisting}[language=Java]
// assume xs.length is a power of 2
int halvingSum(int[] xs) {
    if (xs.length == 1) return xs[0];
    else {
        int[] ys = new int[xs.length/2];
        for (int i=0;i<ys.length;i++)
            ys[i] = xs[2*i]+xs[2*i+1];
        return halvingSum(ys);
    }
}
\end{lstlisting}
\vspace*{4pt}

\textbf{1. Method of Measuring Problem Size:}\\ 
The problem size is measured using the length of the input array \ttt{xs[]}.
\\
    
\textbf{2. Recurrence Relation:}\\ 
Let n = length of the array \ttt{xs[]}. The recurrence relation can be written as:
\begin{align*}
 T(n) &=  \underbrace{T({n\over2})}_{\text{recurrence}} +
\underbrace{({n\over2})\cdot c_1}_{\text{allocating new array}} +
\underbrace{({n\over2})\cdot c_2}_{\text{for loop}}
; T(1) = O(1)\\
T(n) &=  T({n\over2}) +O(n)
 ; T(1) = O(1)\\
\end{align*}

\textbf{3. Running Time Complexity}\\ 
According to a table of common recurrences assuming T(0) and T(1) are constant the reccurence relation solves to the following:

\begin{align*}
    T(n) &= T({n\over2}) +O(n) \rightarrow O(n) 
\end{align*}

\pagebreak

\subsection*{P2}
\begin{lstlisting}[language=Java]
int anotherSum(int[] xs) {
    if (xs.length == 1) return xs[0];
    else {
        int[] ys = Arrays.copyOfRange(xs, 1, xs.length);
        return xs[0]+anotherSum(ys);
    }
}
\end{lstlisting}

\textbf{1. Method of Measuring Problem Size:}\\ 
The problem size is measured using the length of the input array \ttt{xs[]}.
\\
    
\textbf{2. Recurrence Relation:}\\ 
Let n = length of the array \ttt{xs[]}. The recurrence relation can be written as:
\begin{align*}
 T(n) &=  \underbrace{T({n-1})}_{\text{recurrence}} +
\underbrace{({n - 1})\cdot c_1}_{\text{copying array}}
; T(1) = O(1)\\
T(n) &=  T({n-1}) +O(n)
 ; T(1) = O(1)\\
\end{align*}

\textbf{3. Running Time Complexity}\\ 
According to a table of common recurrences assuming T(0) and T(1) are constant the reccurence relation solves to the following:

\begin{align*}
    T(n) &=  T({n-1}) +O(n) \rightarrow O(n) 
\end{align*}


\hrule

\subsection*{P3}
\begin{lstlisting}[language=Java]
int[] prefixSum(int[] xs) {
    if (xs.length == 1) return xs;
    else {
        int n = xs.length;
        int[] left = Arrays.copyOfRange(xs, 0, n/2);
        left = prefixSum(left);
        int[] right = Arrays.copyOfRange(xs, n/2, n);
        right = prefixSum(right);
        int[] ps = new int[xs.length];
        int halfSum = left[left.length-1];
        for (int i=0;i<n/2;i++) { ps[i] = left[i]; }
        for (int i=n/2;i<n;i++) { ps[i] = right[i - n/2] + halfSum; }
        return ps;
    }
}
\end{lstlisting}

\textbf{1. Method of Measuring Problem Size:}\\ 
The problem size is measured using the length of the input array \ttt{xs[]}.
\\
    
\textbf{2. Recurrence Relation:}\\ 
Let n = length of the array \ttt{xs[]}. The recurrence relation can be written as assuming that:
\begin{align*}
 T(n) &=  \underbrace{T({n\over2})+T({n\over2})}_{\text{recurrence left + right}} +
\underbrace{({n\over2})\cdot c_1+({n\over2})\cdot c_1}_{\text{copying array left + right}}
; T(1) = O(1)\\
T(n) &=  2T({n\over2})+O(n)
 ; T(1) = O(1)\\
\end{align*}

\textbf{3. Running Time Complexity}\\ 
According to a table of common recurrences assuming T(0) and T(1) are constant the reccurence relation solves to the following:
\begin{align*}
   T(n) &=  2T({n\over2})+O(n) \rightarrow O(n \cdot log_2n)
   \end{align*}


