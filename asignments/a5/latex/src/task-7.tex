 \chapter{Task 7: Counting Dashes}

\begin{lstlisting}[language=Java]
void printRuler(int n) {
    if (n > 0) {
        printRuler(n-1);
        // print n dashes
        for (int i=0;i<n;i++) System.out.print('-');
        System.out.println();
        // --------------
        printRuler(n-1);
    }
}
\end{lstlisting}


The recurrence relation of the function \ttt{printRuler()} is given as the following:
\[
 g(n) = 2g(n-1) + n;  g(0) = 0
\]
The output of the recurrence gives the total amount of dashes needed to draw a rule of size n. The following shows how the recurrence can be solved with a closed form function.
\begin{align*}
    f(n) &= 2^n - 1\\
    g(n) &= a \cdot f(n) + b\cdot n + c  
\end{align*}
Let me first list out the outputs of the recurrence relation for when n=0, 1, 2 and 3. This sample will give us an idea on how the output grows relation to n. Additionally, while developing the close form of the recurrence it gives us the opportunity to compare the outputs, so we can strategically tweak variables to get the desired outputs.
\begin{align*}
&g(0) = 0\\
&g(1) = 1\\
&g(2) = 4\\
&g(3) = 11
\end{align*}

\textbf{(i):} 
Lets first find c. Let us do what is advised and plug in 0 into g(n).
\begin{alignat*}{3}
    g(0) &= a\cdot f(0) &&+ &&b\cdot 0 + c \\
         &= a\cdot 0 &&+  &&b\cdot 0 + c = 0
\end{alignat*}

\textbf{(ii):} 
Notice that the first 2 terms of the equation $a\cdot 0 + b\cdot 0$ must equal zero. Hence, $c=0$ because $g(0) = 0$ shown in the list of outputs.
Now we need to find the values of a and b. Lets plugin in 3 into g(n).
\begin{alignat*}{3}
    g(3) &= a\cdot f(3) &&+ &&b\cdot 3 + c \\
         &= a\cdot 7 &&+  &&b\cdot 3 + 0 = 11 
\end{alignat*}

\textbf{(iii):} 
We can substitute in f(n) to form the closed for g(n).
\begin{alignat*}{3}
    g(n) &= 2f(n)-n \\
         &= 2(2^n - 1)-n \\  
\end{alignat*}

\pagebreak

\textbf{(iv):} 
Let's prove that the close form is true for all values of n using induction. Let's prove that $g(n) = 2g(n - 1) + n = 2(2^n - 1)-n $
\\\\
\textbf{Predicate:}\\
Assuming that the recurrence relation is $g(n) = 2g(n - 1) + n  = 2f(n) - n$.
\[
  P(n) = \underbrace{2f(n)- n}_{\text{recurrence}}  = \underbrace{2(2^n - 1)-n}_{\text{close form}}
\]
\textbf{Base Case:} 
\[
  P(0) = 0
\]
This base case is true because\dots
\begin{align*}
    P(0) &= \underbrace{2f(0)- 0}_{\text{recurrence}} &&= \underbrace{2(2^0 - 1)-0}_{\text{close form}}\\
         &= 0 &&= 0 &&\rightarrow \text{true}
\end{align*}

\textbf{Inductive Step:}\\ 
Let us assume that $P(k)$ is true for all positive integer values of k. We want to show when $P(k)$ is true then $P(k+1)$ is true also.
\begin{align*}
 g(k+1) &= g(k+1)\\
 2f(k+1) - (k+1) &= 2(2^{k+1}-1) - (k + 1)
\end{align*}
LHS:
\begin{align*}
    &= 2f(k+1) - (k+1)\\
\end{align*}
Let's substitute the given close form of f(n) thus we get\dots
\begin{align*}
    &= 2(2^{k+1}-1) - (k+1)\\
    &=\text{RHS}
\end{align*}
    
Hence, this proves that when P(k) is true P(k+1) is also true. Now using mathematical induction we can conclude that P(n) is true for all positive integer values of n which concludes the proof.   