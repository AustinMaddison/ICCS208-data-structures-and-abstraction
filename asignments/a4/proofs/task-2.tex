\chapter{Task 4: Sum Square}
\begin{thm}[]
	Any $2^n$ by $2^n$
	grid with one painted cell can be tiled using L-shaped triominoes
	such that the entire grid is covered by triominoes but no triominoes overlap with each other
	nor the painted cell. 
	\par
\end{thm}	


\section{Proof by induction.}
\subsection{Predicate}
$P(n)$ is true when a triminoe can fit into a $2^n$ by $2^n$ grid where once cell is painted out.

\subsection{Basis Step}
$P(1)$ is true because for all the possible $2^1$ by $2^1$ grids with a painted out cell, triminoes can tile as shown in Figure \ref{fig:base-cases}. 

\begin{figure}[h]
	\centering
	%\incsvg{path/}{path/file}
	\incsvg{figures}{figures/T1-figure1}{100mm}
	\caption{All possible triminoe placement if a single cell was painted out}
	\label{fig:base-cases}
\end{figure}

\subsection{Inductive Step}
Lets assume that $P(k)$ is true for all positive interger values of k.\\\\
\textbf{Want to show} when $P(k)$ is true, then $P(k+1)$ is true also.
Consider a $P(k+1)$ grid that is subdivided into equal quadrants vertically and horizontally. Notice that each quadrant of $P(k+1)$ is $P(k)$.Then we paint out 1 cell.\\

\begin{figure}[ht]
	\vspace{-10pt}
	\centering
	%\incsvg{path/}{path/file}
	\incsvg{figures}{figures/T1-figure2}{30mm}
	\incsvg{figures}{figures/T1-figure3}{30mm}
	\vspace{-10pt}
	\caption{$P(k+1)$}
	\label{fig:T1-figure2}
\end{figure}

\noindent
 At this stage its not apparent if the triminoes can fill the rest of the grid. We need to somehow introduce cases made in basis step into the quadrants.

\pagebreak
\noindent
Let's paint in the middle cells, only the quadrants that havn't been painted in yet. This achieves two things, firstly this achieves placing a triminoe and secondly each quadrant now has 1 cell painted out. 


% make cooler illustration if time permits
\begin{figure}[H]
	\centering
	%\incsvg{path/}{path/file}
	\incsvg{figures}{figures/T1-figure4}{30mm}
	\caption{Painting out the middle cells.}
	% \vspace{-10pt}
	\label{T1-figure4}
\end{figure}

\noindent
Since all quadrants now satisfy the conditions of $P(k)$ which is having 1 cell painted out and we know from the basis step that every quadrant is tileable we can conclude through inuctive hypothesis that $P(k+1)$ is true too. This completes the inductive step.

\begin{figure}[H]
	\centering
	\vspace{-5pt}
	%\incsvg{path/}{path/file}
	\incsvg{figures}{figures/T1-figure5}{30mm}\\
	\caption{Tiling the quadrants using the basis step cases.}
	\label{T1-figure4}
\end{figure}

\subsection{Conclusion}
Since we can show that when $P(k)$ is true then $P(k+1)$ is true we can conclude by mathmatical induction that $P(n))$ is true for all positive integer values of n. 

















