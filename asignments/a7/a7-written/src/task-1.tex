\chapter{Task 1: Mathematical Truth}

\small \subsection*{Proof by strong induction.}

\textbf{Proposition}

Every binary tree on n nodes where each has either zero or two children had precisely ${n+1\over2}$ leaves.\\

\textbf{Basis Step:}\

\begin{align*}
    P(1) = {{1+1}\over 2} = 1
\end{align*}

This base case is true because a single node can be considered to have no children thus it the single node is a leaf.\\


\textbf{Inductive Step:}\
(I.H) When P(j) for $1 \leq j \leq k$  then P(k+2) is also true. The constant 2 comes from the proposition that any node can only have 2 or no children thus if we were to want to add more nodes to the tree it must be by 2 nodes at a time.\\

To prove that P(j) is true for $1 \leq j \leq k$ we must consider the invariance caused by the const raint. The binary tree nodes can only have 2 or no children. Suppose binary tree as a state machine starting with a single node n=1. The operation for growing a tree is constrained such that to have more nodes we have to add exactly 2 nodes to a leaf. This in turn also increases the amount of leaves by 1.


\includesvg[width= 1.0in]{Graph}

% \begin{figure}[h]
% 	\centering
% 	%\incsvg{path/}{path/file}
% 	\incsvg{figures}{figures/Graph}{100mm}
% 	\includesvg[inkscapeformat=svg]{Graph}
	
% 	% \caption{All possible triminoe placement if a single cell was painted out}
% 	\label{fig:base-cases}
% \end{figure}



