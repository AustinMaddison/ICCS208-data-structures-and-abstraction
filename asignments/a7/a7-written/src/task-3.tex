\chapter{Task 3: Quick Sort Recurrence}

\subsection*{(ii)}
$$g(n) ={f(n)\over n+1 }$$
$$f(n) =2+{(n+1) \cdot f(n-1)\over n } $$
\begin{align*}
    g(n) & ={f(n)\over n+1 } \\[5pt]
         & ={2+{(n+1) \cdot f(n-1)\over n } \over n+1 }\\[5pt]
         & ={2 \over n+1} + {f(n-1) \over n} \\[5pt]
         & ={2 \over n+1} + {g(n-1)} \\[5pt]
\end{align*}
$$ g(n)={2 \over n+1} + {g(n-1)}\\[5pt]$$

\hrule

\subsection*{(iii)}

\begin{align*}
    g(n) & ={2 \over n+1} + {g(n-1)}
\end{align*}

The recurrence expands to:

\begin{align*}
    g(n) & = 
       {2 \over n+1} +\
       {2 \over n} +
       {2 \over n-1} +
       {2 \over n-2} +
       {2 \over n-3} +
       \dots +
       {2 \over 2}
\end{align*}

\begin{align*}
    g(n) & = 2 (
       {1 \over n+1} +\
       \underbrace{
       {1 \over n} +
       {1 \over n-1} +
       {1 \over n-2} +
       {1 \over n-3} +
       \dots +
       {1 \over 2}
       }_{h(n) - 1}
    )
\end{align*}
We can substitute the series with the function $h(n)-1$ harmonic numbers.

\begin{align*}
    g(n) & = 2 
    (
       {1 \over n+1} +
       h(n) -1
    )
\end{align*}

This completes the form of g(n) as there is no longer a recurrence relation.

\subsection*{(iv)}

Lets first write f(n) in terms of n:

\begin{align*}
   f(n) & = g(n) \cdot (n+1)\\[5pt]
   f(n) & = 2({1 \over n+1} + \underbrace{h(n)}_{ln(n)+ 1 \text{ from fact given.}} - 1) \cdot (n+1)\\[5pt]
   f(n) & = 2({1 \over n+1} + (ln(n)+1) - 1) \cdot (n+1)\\[5pt]
   f(n) & = 2({1 \over n+1} + ln(n)) \cdot (n+1)\\[5pt]
\end{align*}

We \textbf{want to show} that $f(n)$ is $O(nln(n))$. This can be solved by the following limit:

\begin{align*}
    & = \lim_{n\to\infty} {f(n) \over nln(n)}\\[5pt]
    & = \lim_{n\to\infty} {2 ({1 \over n+1} + ln(n) \cdot (n+1))   \over nln(n)}\\[5pt]
    & = \lim_{n\to\infty} { 2 + 2ln(n) \cdot (n+1)
                        \over 
                        nln(n)}\\[5pt]
    & = \lim_{n\to\infty} { 2 
                        \over 
                        nln(n)} +
        \lim_{n\to\infty} {2ln(n) \cdot (n+1)
                        \over 
                        nln(n)}\\[5pt]
    & = \lim_{n\to\infty} { 2 
                        \over 
                        nln(n)} +
        \lim_{n\to\infty} {2n + 2
                        \over 
                        n}\\[5pt]
    & = \lim_{n\to\infty} { 2 
                        \over 
                        nln(n)} +
        \lim_{n\to\infty} 2 + { 2
                        \over 
                        n}\\[5pt]
    & = 2
\end{align*}

Notice when n approaches positive infinity the constant 2 is the only term remaining because other terms converge to 0. This shows that the function nln(n) times some constant can over-bound the function f(n) thus proving that f(n) is O(nln(n)).